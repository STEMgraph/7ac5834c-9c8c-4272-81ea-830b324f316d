\learningobjective{At the end of this challenge, the scholar will be able to install and perform fundamental navigation in Vim, including opening, editing, and saving files.}

\begin{challenge}

\chatitle{Getting Started with Vim}

\begin{chadescription}
Vim is a powerful and versatile text editor that is widely used for software development, system administration, and other technical tasks. Mastering Vim can significantly enhance productivity, as it allows for rapid text editing and navigation with minimal reliance on the mouse.

In this challenge, you will learn how to install Vim using your preferred package manager and practice some fundamental commands. The goal is to help you gain confidence in opening, editing, and saving files within Vim. Learning Vim is a foundational skill for anyone working in the command-line environment and provides a stepping stone for mastering more advanced text-editing techniques.

\textbf{Note:} This challenge assumes no prior knowledge of Vim. Make sure you have access to a command-line interface on your machine before starting.
\end{chadescription}

\begin{task}
The first step is to install Vim on your local machine. Vim is available for most platforms, and the installation process involves using your preferred package manager.

\begin{enumerate}
    \item Open your terminal or command-line interface.
    \item Use your preferred package manager to install Vim:
        \begin{itemize}
            \item On Debian/Ubuntu-based systems: Use `sudo apt update && sudo apt install vim`.
            \item On Red Hat/Fedora-based systems: Use `sudo dnf install vim`.
            \item On macOS: Use `brew install vim` if Homebrew is installed.
            \item On Windows: Install Vim using a package manager like `scoop` (`scoop install vim`) or `choco` (`choco install vim`).
        \end{itemize}
    \item Verify the installation by typing `vim --version` in the terminal. You should see the version information displayed.
\end{enumerate}
\end{task}

\begin{task}
Now that Vim is installed, it’s time to explore basic navigation and file editing.

\begin{enumerate}
    \item Open Vim by typing `vim` in the terminal.
    \item To create a new file or open an existing one, use `vim <filename>` (replace `<filename>` with the name of your file).
    \item Learn and practice the following basic commands:
        \begin{itemize}
            \item Enter insert mode to edit text: Press `i`.
            \item Exit insert mode: Press `Esc`.
            \item Save changes: Type `:w` and press `Enter`.
            \item Exit Vim: Type `:q` and press `Enter`.
            \item Save and exit: Type `:wq` and press `Enter`.
            \item Quit without saving: Type `:q!` and press `Enter`.
        \end{itemize}
    \item Experiment with moving the cursor using the arrow keys or `h`, `j`, `k`, `l`.
    \item Close Vim when done.
\end{enumerate}
\end{task}

\begin{questions}
    \item How can you verify that Vim is installed successfully on your system?
    \item What is the difference between `:w`, `:q`, and `:wq` commands in Vim?
    \item What is the purpose of insert mode, and how do you activate it?
    \item What are the two ways you can quit Vim without saving changes?
    \item What are the keys for moving the cursor in normal mode?
\end{questions}

\begin{advise}
To deepen your understanding of Vim, run the command `vimtutor` in your terminal. It provides a hands-on tutorial that covers more advanced features and will solidify your basic knowledge of this powerful text editor.
\end{advise}

\end{challenge}
