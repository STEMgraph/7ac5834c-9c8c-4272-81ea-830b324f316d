%% This is the challenge based learning template. These Challenges are meant to be selfexplanatory and atomic.
%% A scholar should be able to complete a challenge with only the help, that is provided in the tasks.
%% The challenge-sheet consists of an objective, a description, a set of tasks with steps and questions, and additionally sometimes an advise at the end.
%% 
%% All of these challenges are part of the STEMgraph project and will be added to the graph-database, maintained by the STEMgraph team.
%% Each challenge will be assigned a unique ID (generated by uuidgen) and will be stored on www.github.com/STEMgraph/
%% Every challenge is supposed to only teach a single skill. This way, challenges can be reused, mixed and matched.
%% The challenge is written in a LaTeX-like format and can be parsed by a python script to 
%% * generate a PDF
%% * generate a website
%% * generate a REST-API answer
%%

%% The learning objective should state in one sentence what the scholar will be able to do at the end of the challenge. 
%% It shall not include any additional LaTeX commands.
\learningobjective{At the end of this challenge, the scholar will be able to use/perform/do  ...... }

%% The complete content of the challenge is written between the begin and end tags.
%% The challenge can contain every valid LaTeX command, like figure, lstlisting, table, etc.
\begin{challenge}

%% chatitle is the name of the challenge. It should not include any additional LaTeX commands.
%% chatitle will also be parsed by other tools. It is NOT the title of the repo. 
%% The name of a challenge might change over time, but the reference to the UUID will always remain the same.
    \chatitle{Basic Usage of the Challenge-Based-Learning Sheets for STEMgraph} 

%% chadescription is the description of the challenge.
%% It contains a precise introduction of at least 300 words, to explain what the scholar will learn by completing the challenge.
%% In order for it to make sense, it should also contain information about how the learned material is used in day-to-day life. 
    \begin{chadescription}
        Challenges are a great way to learn new skills. 
        If they are selfexplanatory and atomic, they can be used by scholars in a wide range of contexts, and repeated over and over again.
        In comparision to learning materials (like books or longer tutorials), these challenges are meant to be used by teachers, to arrange new paths through a graph of skills.

        \begin{figure}
            \centering
            \includegraphics[width=0.9\textwidth]{./screenshot.png}
            \caption{Screenshot of the STEMgraph node-network}
        \end{figure}

        \textbf{Note: }Sometimes there needs to be a little bit of a warning or a hint for the scholar, before starting the challenge. 
        This should go here!
    \end{chadescription}

%% A challenge consists of a set of tasks and questions.
%% The goal of the tasks is, to let a scholar complete certain actions, almost blindly by following the instructions.
%% The questions are meant to provoke a deeper understanding of the material.
%% Nontheless, before the instructions start, there should be a little bit of context to explain the task and its background.
    \begin{task}
        Generating a new challenge is not that hard! 
        Try to think of something that you would like to learn. 
        If you can teach it to someone in less than 30 minutes by explaining it, it looks like a great challenge to me.
        Often times, these challenges will only teach a single command or skill, or a few variations of it.
        The goal here is to not overload the scholar with too many new concepts. 
        In order to become good at STEM, it is important to learn a single skill at a time and repeat it until the foundations are really solid!
        It also isn't a problem, if the task explained in too small steps. 
        Of course, it shouldn't be boring by being repetetive, but it should be slow and detailed.
        Always make sure, to expect the minmal amount of prerequisites from the scholar. 
        Only this way the STEMgraph can ensure, that challenges are mix-and-match-able.
%% It is not necessary for every task to have steps in the enumeration style. 
%% Sometimes it can be more useful to just write the task in one paragraph.
%% In any case, the task should be structured carefully, not leaving out a single step.
%% Think of the "making-a-sandwich-manual" experiment to come up with a good description of the task.
        \begin{enumerate}
            \item Write down a skill, on a piece of paper that you really like.
            \item Come up with a good explaination, why this skill is useful in STEM. 
            \item Explain, why this is a basic task to perform.
            \item Think of 3 to 4 applications of this skill.
            \item Answer the questions below.
            \item Write a request to ChatGPT using the following prompt:
            \begin{lstlisting}
                I am working on github.com/STEMgraph challenges.
                Today I worked on the following:
                ```
                <insert raw text of this challenge here>
                ```
                Please give me detailed feedback to my answers to the provided questions.
                Here are my answers in the correct order:
                ```
                <insert your answers to the tasks questions here>
                ```
            \end{lstlisting}
        \end{enumerate}

%% The questions should be structured carefully and don't ask to much information at once. 
%% Especially when a Chatbot should be able to correct the scholars answers, the questions should be precise and concise.
%% Have at least three questions to every task.
        \begin{questions}
            \item What is the skill you want to teach your scholars?
            \item Why do you think its a foundationary skill?
            \item What are the applications of this skill?
            \item What prerequisites do you need to perform this skill?
            \item What is the first step to perform this skill?
            \item What advice do you have for the scholar at the end of this challenge?
        \end{questions}
    \end{task}

%% If you want the scholar to take some further learning from the tasks that they just performed, use the advice tag.
%% The advise should not be longer than maybe 80 words.
    \begin{advice}
        Let someone from your peer-group try out your challenges!
        Have them generate issues in the github repository and let them write down how long it took them to complete the challenge.
        A good challenge is one, that takes less than 30 minutes to complete.
        This way, scholars can plan their day.
        Happy generating!
    \end{advice}
\end{challenge}
